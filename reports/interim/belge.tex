
\documentclass[12pt,a4paper]{article}

% Türkçe %
\usepackage[utf8]{inputenc} %Türkçe karakterler için
\usepackage[T1]{fontenc}
\renewcommand{\tablename}{Tablo}
\renewcommand{\figurename}{Şekil}
\renewcommand{\indexname}{Dizin}
\renewcommand{\listfigurename}{Şekiller}
\renewcommand{\listtablename}{Tablolar}
\renewcommand{\contentsname}{İçindekiler}
\renewcommand{\refname}{Kaynaklar}
% Türkçe %

\usepackage{pdfpages}
\usepackage{geometry}
\usepackage{graphicx} %Resim koymak için
\usepackage{times} %Times fontu
\usepackage{graphicx}
\usepackage[nottoc]{tocbibind}
\usepackage{url}

\setcounter{tocdepth}{4}
\setcounter{secnumdepth}{4}

\begin{document}

    \pagenumbering{gobble}
    \renewcommand{\abstractname}{Özet}
\begin{titlepage}
    \begin{center}
        \begin{large}
            \vspace*{0.5cm}
            GAZİ ÜNİVERSİTESİ \\
            MÜHENDİSLİK FAKÜLTESİ \\
            BİLGİSAYAR MÜHENDİSLİĞİ

            \vfill
            BM495 BİLGİSAYAR PROJESİ I \\
            ARA RAPOR
            
            \vfill
            Yapay Öğrenme Algoritmalarının Uygulanması

            \vfill
            \begin{abstract}
                Bu belgede, proje başlangıcından bugüne kadar kat edilen merhaleler
                beyan edilmiştir. Kullanılacak araçlar ve yöntemler açıklanmış,
                bu araç ve yöntemlerin kullanımlarına dair yapılmış olan çalışmalar
                amaç ve kazanımlarıyla beraber sunulmuştur.
            \end{abstract}
            \vfill
            Abdullah Akalın\\Karim El Guermai\\Muhammed Emre Emrah\\

            \vfill
            \vspace{0.5cm}
            23.11.2017
        \end{large}
    \end{center}
\end{titlepage}

    \newpage

    \pagenumbering{roman}
    \tableofcontents
    \newpage

    \pagenumbering{arabic}

    \section{Projenin Gerçekleştirilmesi}
    \subsection{Proje Hedefi}
    Bu projenin hedefi derin öğrenme yöntemleri kullanarak görüntü verilerinde nesne tanımaya yönelik problem çözmektir.

    \subsection{Projede Mevcut Aşama}
    Sunulan SPMP belgesindeki proje yönetim planında (bkz. SPMP, '3.1.2 Verinin Elde Edilmesi ve Hazırlanması') beyan edilmiş olan 
    'Derin Öğrenme Yöntemlerinin Araştırılması' görevi tamamlanmış
    olup, 'Verinin Elde Edilmesi ve Hazırlanması' aşamasına geçilmiştir. Bu görevde amaç, projenin amacını gerçekleştirmeye uygun ve aynı zamanda projenin
    konusunu belirleyecek verinin bulunmasıdır.

    \subsubsection{Arayüz}
    Projenin çıktısında Python konsolu haricinde bir arayüz bulunmayacaktır.

    \paragraph{Kullanılan Teknolojiler}
    Python konsolu ve Jupyter Notebook web arayüzü kullanılmaktadır.

    \paragraph{Tasarımsal Olarak Kullanılması Planlanmış, İhtiyaç Duyulan Teknolojiler}
    Herhangi bir arayüz tasarımı yapılmayacaktır.
    
    \subsubsection{Yazılım Sistemi}

    \paragraph{Kullanılan Teknolojiler} \label{tech}
    Bu projede programlama dili olarak python kullanılacak olup, derin öğrenme için ise python için yazılmış Keras modülü kullanılacaktır.
    Keras modülü arkaplanda yine bir python modülü olan Tensorflow kütüphanesini kullanmaktadır. Bunların yanı sıra yapay öğrenmede sıkça kullanılan
    NumPy, Pandas ve scikit-learn modüllerinden yararlanılmaktadır.
    Geliştirme ve raporlama için Jupyter Notebook aracı kullanılmaktadır.

    \paragraph{Tasarımsal Olarak Kullanılması PLanlanmış, İhtiyaç Duyulan Teknolojiler}
    Bölüm \ref{tech} kısmında belirtilen teknolojiler kullanılacak ana teknolojiler olup, bunların haricinde kullanılması gerekebilecek teknolojiler
    henüz belli değildir.

    \subsection{Projeye Dair Planlar}
    Projenin istenilen başarıya erişmesi durumunda, arayüz tasarımı gerçekleştirilmesi planlanmaktadır.
    Bu durumda eğitimli model keyfi veriler üzerinde, amaca yönelik işlevler gerçekleştirebilecektir.

    \subsubsection{Kullanılması Planlanan Diğer Teknolojiler}
    Sahip olunan bilgisayar kaynaklarının derin öğrenme modelinin eğitilmesinde yetersiz kalması durumunda imkan dahilinde olan bulut çözümlerinin
    araştırılması planlanmaktadır.
    
    \section{Durum Değerlendirilmesi}

    \subsection{Teslim ve Temeltaşlar}
    Aşağıda, projenin şimdiye kadar kat ettiği merhaleler ayrıntılı olarak ele alınmıştır.

    \subsubsection{Kaynak Araştırması}
    \paragraph{Açıklama}
    Bu aşamada yapay öğrenmeye dair araştırmalar gerçekleştirilmiştir. İlgili kaynaklar taranmış ve yapay öğrenmenin uygulama alanları araştırılmıştır.
    Stanford Üniversitesi'ne ait yapay öğrenme projelerinin ayrıntılı listesine ulaşılıp, yapılmış olan projeler etraflıca tetkik edilmiştir.

    \paragraph{Amaç}
    Bu görevin amacı, ne türlü projeler yapılabileceğini görüp, istikametimizi tayin etmek olmuştur. 
    Ayrıca bu sahada kullanılan terminolojiye aşinalık kazanmak hedeflenmiştir.

    \paragraph{Kazanımlar}
    Yapılan araştırmalar neticesinde, yapay öğrenmenin ses tanımadan, yüz tanımaya pek çok işte kullanıldığı öğrenilmiş,
     görüntü işlemedeki kullanım alanı dikkatimizi celbetmiştir. Grubumuz, danışman tarafından araştırma alanını
     daraltmaya yönlendirilmiştir. Ayrıca grubumuz, yine danışman tarafından, derin öğrenme konusunu araştırmaya teşvik edilmiştir.

    \subsubsection{Proje İncelemeleri}
    \paragraph{Açıklama}
    Bu aşamada, bir önceki adımda incelenen projelere ilave olarak derin öğrenme ve derin öğrenme projeleri araştırılmıştır. Her üye kendince beğendiği iki projeyi ele almış
    ve ayrıntılı araştırmıştır. Yapılan araştırmalar raporlanıp danışmana sunulmuştur\footnote{https://github.com/emremrah/deep-learning/blob/master/reports/arastirma/belge.pdf}.

    \paragraph{Amaç}
    Bu çalışmamızdaki amaç, genelden özele geçiş yapmak, derin öğrenmeyi araştırmak ve bu alandaki projeleri incelemek olmuştur. Herbir üye ikişer proje ele almış, toplam altı proje incelenmiştir. 
    İncelenen projelerde kullanılan metodların irdelenmesi de bu adımın amaçlarından olmuştur.

    \paragraph{Kazanımlar}
    Seçilen projelerinin beşinin görüntü işlemeye, birinin doğal dil işlemeye ait olduğu görülmüştür. Bu sonuçtan yola çıkılarak, görüntü işlemeye dair bir proje geliştirme kararı alınmıştır.
    Grubumuz bu aşamada derin öğrenme uygulamaları geliştirmek için platform seçmeye yönlendirilmiştir. Ayrıca giriş alıştırması mahiyetinde bir uygulama geliştirilmesi istenmiştir.
    Grubumuz gerekli araştırmaları yapmış ve Python programlama dili ile bu dile ait Keras derin öğrenme kütüphanesini kullanmaya karar vermiştir.

    \subsubsection{Keras ile MNIST Verileri Üzerinde Derin Öğrenme Alıştırması}
    \paragraph{Açıklama}
    Bu adımda Keras kütüphanesi kullanılarak eğitim için yaygın bir şekilde kullanılan MNIST el yazısı veri seti ile bir derin öğrenme uygulaması gerçekleştirilmiştir. 
    Uygulama Python programlama dili ile Jupyter Notebook yazılımı üzerinde geliştirilmiş olup internet üzerinden paylaşılmıştır.\footnote{https://github.com/emremrah/deep-learning/blob/master/src/mnist\_tutorial/keras\_tutorial.ipynb}
    \paragraph{Amaç}
    Keras kütüphanesinin kullanım şeklini öğrenmek, bir derin öğrenme modelinin nasıl oluşturulduğunu öğrenmek ana gayemiz olmuştur. Ayrıca verilerin modellenmeden önce geçmesi gereken önişlemlerin
    öğrenilmesi istenilmiştir.

    \paragraph{Kazanımlar}
    Derin öğrenmenin uygulanmasına dair tecrübe elde edilmiştir. Derin öğrenmede kullanılan terminoloji hakkında bilgi edinilmiştir. Bu adımdan sonra grubumuz derin öğrenme ile nesne tanıma alıştırması
    yapmaya teşvik edilmiştir.

    \subsubsection{Keras ile CIFAR-10 Verileri Üzerinde Nesne Sınıflandırma Alıştırması}
    \paragraph{Açıklama}
    Keras ile CIFAR-10 verilerini kullanarak nesne tanıma uygulaması gerçekleştirilmiştir.
    Yapılan uygulama yine Jupyter Notebook formatında internette yayınlanmıştır.\footnote{https://github.com/emremrah/deep-learning/blob/master/src/cifar\_10\_tutorial/cifar\_10\_tutor.ipynb}

    \paragraph{Amaç}
    Bu çalışmanın amacı Keras kütüphanesini kullanarak nesne tanıma problemine uygun modellerin nasıl geliştirileceği hakkında bilgi sahibi olmak olmuştur.
    Model performansını etkileyen amiller öğrenilmek istenmiştir. Ayrıca görüntü verilerine has önişlemlerin öğrenilmesi hedeflenmiştir.

    \paragraph{Kazanımlar}
    Bu çalışmanın sonucunda nesne tanıma probleminin yüksek bir bilgisayar gücü gerektirdiği anlaşılmıştır.
    Modelin derinleştirilmesi ile performansın artırılabileceği gözlenmiş olsa da, tecrübenin burada büyük rol oynadığı anlaşılmıştır.
    Keras kütüphanesini kullanmada bir nebze daha yetkinlik kazanılmıştır.

    \subsection{Kaynak ve İhtiyaçlar}
    Projede bulunulan merhalede daha önce belirtilen kaynak ve ihtiyaçlardan başka gereksinim bulunmamaktadır. Ancak sahip olunan bilgisayar gücünün yetersiz
    kalması durumunda yeni ihtiyaçlar doğması muhtemeldir.

    \subsection{Bağımlılık Değişiklikleri}
    Bağımlılıklarda değişiklik olmamıştır.

    \subsection{Gereksinim Değişiklikleri}
    Gereksinimlerde değişiklik olmamıştır.

\end{document}


