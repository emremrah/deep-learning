
\documentclass[12pt,a4paper]{article}

% Türkçe %
\usepackage[utf8]{inputenc} %Türkçe karakterler için
\usepackage[T1]{fontenc}
\renewcommand{\tablename}{Tablo}
\renewcommand{\figurename}{Şekil}
\renewcommand{\indexname}{Dizin}
\renewcommand{\listfigurename}{Şekiller}
\renewcommand{\listtablename}{Tablolar}
\renewcommand{\contentsname}{İçindekiler}
\renewcommand{\refname}{Kaynaklar}
% Türkçe %

\usepackage{pdfpages}
\usepackage{geometry}
\usepackage{graphicx} %Resim koymak için
\usepackage{times} %Times fontu
\usepackage{graphicx}
\usepackage[nottoc]{tocbibind}
\usepackage{url}

\begin{document}

    \pagenumbering{gobble}
    \begin{titlepage}
   \begin{center}
      \begin{large}
         \vspace*{0.5cm}
         GAZİ ÜNİVERSİTESİ \\
         MÜHENDİSLİK FAKÜLTESİ \\
         BİLGİSAYAR MÜHENDİSLİĞİ

         \vfill
         BM495  BİLGİSAYAR PROJESİ \\
         SPMP BELGESİ

         \vfill
         Abdullah Akalın\\Karim El Guermai\\Muhammed Emre Emrah\\

         \vfill
         \vspace{0.5cm}
         01.11.2017
      \end{large}
   \end{center}
\end{titlepage}

    \newpage

    \pagenumbering{roman}
    \tableofcontents
    \newpage

    \pagenumbering{arabic}

    \section{Giriş}
    Bu belge, proje grubumuzun şimdiye değin yaptığı çalışma ve araştırmaları hülasa etmektedir.
    Projemize dair sair teferruat daha önce sunulmuş olan SPMP\footnote{https://github.com/emremrah/deep-learning/blob/master/reports/spmp/spmp.pdf} ve SRS\footnote{https://github.com/emremrah/deep-learning/blob/master/reports/srs/srs.pdf} raporlarında mevcut olup,
    ihtiyaç halinde müracaat edilmesi rica olunur. 

    Belgenin devamında yapılan çalışmalar başlıklar halinde sıralanmış, her bir çalışmanın teferruatı beyan edilmiştir. Sonuç kısmında ise, mevcut durum değerlendirilip, sonraki çalışmalarımız için yol haritası belirlenmeye gayret edilmiştir.

    \section{Kullanılacak Araç ve Yöntemler}
    Yapılacak olan derin öğrenme projesi için Python dili ve Keras kütüphanesinin kullanılmasına karar verilmiştir.
    Bu kararların veriliş süreci \ref{yap} numaralı bölümde adım adım işlenmiştir. Ayrıca raporlanma ve sunum kolaylığı
    sunması, yapay öğrenmede sıklıkla tercih edilmesi sebebiyle Jupyter interaktif web uygulaması\footnote{http://jupyter.org/} geliştirme ve sunum ortamı olarak
    kullanılacaktır.

    \section{Yapılan Çalışmalar} \label{yap}
    \subsection{Kaynak Araştırması}
    \subsubsection{Açıklama}
    Bu aşamada yapay öğrenmeye dair araştırmalar gerçekleştirilmiştir. İlgili kaynaklar taranmış ve yapay öğrenmenin uygulama alanları araştırılmıştır.
    Stanford Üniversitesi'ne ait yapay öğrenme projelerinin ayrıntılı listesine ulaşılıp, yapılmış olan projeler etraflıca tetkik edilmiştir.

    \subsubsection{Amaç}
    Bu görevin amacı, ne türlü projeler yapılabileceğini görüp, istikametimizi tayin etmek olmuştur. 
    Ayrıca bu sahada kullanılan terminolojiye aşinalık kazanmak hedeflenmiştir.

    \subsubsection{Kazanımlar}
    Yapılan araştırmalar neticesinde, yapay öğrenmenin ses tanımadan, yüz tanımaya pek çok işte kullanıldığı öğrenilmiş,
     görüntü işlemedeki kullanım alanı dikkatimizi celbetmiştir. Grubumuz, danışman tarafından araştırma alanını
     daraltmaya yönlendirilmiştir. Ayrıca grubumuz, yine danışman tarafından, derin öğrenme konusunu araştırmaya teşvik edilmiştir.

    \subsection{Proje İncelemeleri}
    \subsubsection{Açıklama}
    Bu aşamada, bir önceki adımda incelenen projelere ilave olarak derin öğrenme ve derin öğrenme projeleri araştırılmıştır. Her üye kendince beğendiği iki projeyi ele almış
    ve ayrıntılı araştırmıştır. Yapılan araştırmalar raporlanıp danışmana sunulmuştur\footnote{https://github.com/emremrah/deep-learning/blob/master/reports/arastirma/belge.pdf}.

    \subsubsection{Amaç}
    Bu çalışmamızdaki amaç, genelden özele geçiş yapmak, derin öğrenmeyi araştırmak ve bu alandaki projeleri incelemek olmuştur. Herbir üye ikişer proje ele almış, toplam altı proje incelenmiştir. 
    İncelenen projelerde kullanılan metodların irdelenmesi de bu adımın amaçlarından olmuştur.

    \subsubsection{Kazanımlar}
    Seçilen projelerinin beşinin görüntü işlemeye, birinin doğal dil işlemeye ait olduğu görülmüştür. Bu sonuçtan yola çıkılarak, görüntü işlemeye dair bir proje geliştirme kararı alınmıştır.
    Grubumuz bu aşamada derin öğrenme uygulamaları geliştirmek için platform seçmeye yönlendirilmiştir. Ayrıca giriş alıştırması mahiyetinde bir uygulama geliştirilmesi istenmiştir.
    Grubumuz gerekli araştırmaları yapmış ve Python programlama dili ile bu dile ait Keras derin öğrenme kütüphanesini kullanmaya karar vermiştir.

    \subsection{Keras ile MNIST Verileri Üzerinde Derin Öğrenme Alıştırması}
    \subsubsection{Açıklama}
    Bu adımda Keras kütüphanesi kullanılarak eğitim için yaygın bir şekilde kullanılan MNIST el yazısı veri seti ile bir derin öğrenme uygulaması gerçekleştirilmiştir. 
    Uygulama Python programlama dili ile Jupyter Notebook yazılımı üzerinde geliştirilmiş olup internet üzerinden paylaşılmıştır.\footnote{https://github.com/emremrah/deep-learning/blob/master/src/mnist\_tutorial/keras\_tutorial.ipynb}
    \subsubsection{Amaç}
    Keras kütüphanesinin kullanım şeklini öğrenmek, bir derin öğrenme modelinin nasıl oluşturulduğunu öğrenmek ana gayemiz olmuştur. Ayrıca verilerin modellenmeden önce geçmesi gereken önişlemlerin
    öğrenilmesi istenilmiştir.

    \subsubsection{Kazanımlar}
    Derin öğrenmenin uygulanmasına dair tecrübe elde edilmiştir. Derin öğrenmede kullanılan terminoloji hakkında bilgi edinilmiştir. Bu adımdan sonra grubumuz derin öğrenme ile nesne tanıma alıştırması
    yapmaya teşvik edilmiştir.

    \subsection{Keras ile CIFAR-10 Verileri Üzerinde Nesne Sınıflandırma Alıştırması}
    \subsubsection{Açıklama}
    Keras ile CIFAR-10 verilerini kullanarak nesne tanıma uygulaması gerçekleştirilmiştir.
    Yapılan uygulama yine Jupyter Notebook formatında internette yayınlanmıştır.\footnote{https://github.com/emremrah/deep-learning/blob/master/src/cifar\_10\_tutorial/cifar\_10\_tutor.ipynb}

    \subsubsection{Amaç}
    Bu çalışmanın amacı Keras kütüphanesini kullanarak nesne tanıma problemine uygun modellerin nasıl geliştirileceği hakkında bilgi sahibi olmak olmuştur.
    Model performansını etkileyen amiller öğrenilmek istenmiştir. Ayrıca görüntü verilerine has önişlemlerin öğrenilmesi hedeflenmiştir.

    \subsubsection{Kazanımlar}
    Bu çalışmanın sonucunda nesne tanıma probleminin yüksek bir bilgisayar gücü gerektirdiği anlaşılmıştır.
    Modelin derinleştirilmesi ile performansın artırılabileceği gözlenmiş olsa da, tecrübenin burada büyük rol oynadığı anlaşılmıştır.
    Keras kütüphanesini kullanmada bir nebze daha yetkinlik kazanılmıştır.

    \newpage

    \section{Sonuç ve Değerlendirme}
    Yaptığımız tüm bu çalışmalar neticesinde, daha önce çok az bir fikir sahibi olduğumuz yapay ve derin öğrenme konularına aşina olunmuştur. Bir derin öğrenme uygulamasının
    nasıl geliştirildiği öğrenilmiştir ve gerekli araçları edinilmiştir. Sonraki aşama, kendi problemimizi belirleyip, danışmanımızın yönlendirmesiyle ve bu çalışmalardan öğrenilenlerin
    ışığında çalışmalarımıza devam etmektir.

\end{document}


